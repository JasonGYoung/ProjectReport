\chapter{Future Work}
Code generation using TEBNF creates several areas for future work.
\begin{itemize}
  \item Additional I/O methods.  Some of these could include support for TCP/IP, MySQL databases, and others.
  \item A TEBNF Integrated Development Environment (IDE).  The IDE could provide intelligent code completion or incorporate a drag-and-drop interface for adding TEBNF elements to a grammar.
  \item Aspect oriented code generation.  This could involve generating code that leverages aspects.  It could also involve the integration of aspects into the specification of TEBNF itself.
  \item Exploring the use of template meta-programming in generated C++ code.  This programming paradigm emphasizes the use of types, similar to TEBNF.
  \item Software Mining for Graphical User Interface (GUI) code generation \cite{kennard_01,kennard_02}.  TEBNF lends itself to automatic generation of GUI code because of its merging of different input specifications into one.   Kennard proposed the usage of a technique called runtime data mining to generate GUIs \cite{kennard_01}.  Software mining \cite{kennard_02} is a form of data mining that focuses on the inspection of software information characteristics:
  \item Static characteristics: e.g. source code files and database schemas.
  \item Runtime characteristics: e.g. polymorphic data-types, other data values, reading and modification of an instantiated object’s current state.
\end{itemize}