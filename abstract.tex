%
%  (last_modified) Thu Oct 11 08:28:25 2007 by Scott Budge <scott@goga.ece.usu.edu>
%
%  Info: $Id: abstract.tex 31 2007-10-11 14:52:59Z scott $   USU
%  Revision: $Rev: 31 $
% $LastChangedDate: 2007-10-11 08:52:59 -0600 (Thu, 11 Oct 2007) $
% $LastChangedBy: scott $
%

\begin{abstract}
% A space is needed before the text starts so that the first paragraph
% is indented properly.

Errors and inconsistencies between code components can be very costly in a software project.  Efforts to reduce these costs can include the use of tools that limit human interaction with code by generating it from a description.  Most of these tools only generate one of the many components found in a typical application, reducing human-introduced errors within code.   This paper introduces two new works: (1) an input specification called Typed EBNF (TEBNF), and (2) a prototype tool that demonstrates how TEBNF can be used to generate code.  The tool generates code for a console application as described by a TEBNF grammar.  An application built from the generated code will be able to receive input data, parse it, and process it.

  

\end{abstract}


% Local Variables:
% TeX-master: "newhead"
% End:
